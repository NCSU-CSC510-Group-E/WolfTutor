\subsubsection{Further Evaluation}
\label{sec:further-evaluation}
The team has made a best effort to evaluate the algorithm presented in this
paper both using a manually-labeled data set and using in-person interviews with
potential users. That said, true evaluation of educational software must be done
in a classroom setting.  

% TODO: describe ideal study
A good way to evaluate the usefullness of this system would be to truly put it
in the hands of students over a period of time. Over the course of several
semesters, a different set of sections of one course should be evaluated.  For
this reason, it may make sense to use a popular core course to a given major,
such as Intro to Computer Science.  These sections could be grouped into 3
groups, one with no intervention, one where students are pushed to use the
university-sponsored tutoring facilities that exist today, and one where
students are given access to the same facilities but are also given access to
WolfTutor.  Interviews would need to be conducted after student tutoring
sessions throughout the semester asking the students for feedback about the
quality of their tutoring session specifically and features of the application
more broadly.  At the end of the semester(s), the grades of the different groups
could be compared to historical data for the course and some generalizations
could be made about the usefulness of the software.

It is important to note, however, that an improvement in grades may not be the
right measure of success for this system. While better grades would certainly be
ideal, simply higher rates of engagement between students and tutors would be
one reasonable measure of success. Also, reducing time spent scheduling tutoring
sessions between students in the traditional tutoring section and WolfTutor
students would also be a reasonable win for students.  
% TODO: another method of evaluation is asking students to rate their tutors and
% seeing if they are consistently rating their tutors highly or poorly.

\subsubsection{Future Enhancement}
% TODO: describe the process of implementing a new algorithm in this setting
\label{sec:future-enhancement}
The other primary direction for future work is in the application itself.  The
recommendation module has been developed in isolation from the rest of the code
base, so it could easily be plugged into another process with minimal effort.
Whatever module simply needs to implement a method that takes in a list of
tutors and most of the integration work should be quite minimal.  The weighted
average that has already been implemented in this paper can serve as a good
baseline for future evaluation of other methods of classifying tutors.  Other
algorithms that might be a good fit for this problem are clustering algorithms
or classification algorithms.

Clustering algorithms may be a good fit because they all amount to identifying
groupings of similar entries in the population provided.  This is not unlike
what we want to do when searching for a tutor, so it might be possible to
cluster students and suggest tutors that exist in the same clusters as students.  

It is also possible that simple classification algorithms like logistic
regression on the simple end and SVM on the more complex end could be used to
classify tutors as good or bad.  That said, performance is a major concern in a
real-time application such as a chatbot, and such heavy algorithms as clustering
and classification may be too complex to run reliably in a production
environment.  

Another direction for future work is identifying more dimensions on which to
match tutors.  Currently we are matching on three review-related scores and one
academic performance score (gpa).  This could easily be expanded by asking
tutors and students more questions about what they are looking for.  Does the
student prefer test-taking or projects?  Do they speak English natively or would
they prefer to use some other language?  The team intended to implement more
dimensions to match on during the project, but time constraints forced limiting
it to four.  Fortunately, though, the application can easily be extended and
adding more dimensions is a fairly easy future direction.

%%% Local Variables:
%%% mode: latex
%%% TeX-master: "../main"
%%% End:
