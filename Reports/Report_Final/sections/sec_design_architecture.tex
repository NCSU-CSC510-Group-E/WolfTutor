The main architecture of the chatbot was broken up between the ways the user
interacted with it and the helper functionality to support those interactions.  

% Main.js file
Any interaction with the application via text is handled in the ``main.js''
module. In this module are event listeners for text entry events in the Slack
chatbot.  Any time a student is typing a response or initiating contact with the
bot for the first time, the events for handling their inputs are in this file in
a series of listeners and callbacks.  Interestingly, the Botkit API used for
this project does not natively support a more modern JavasScript promise-based
design.

% Dialogs and Prompts
Dialogs are the other main way that students interact with the system.  In
Slack, a dialogue is a pop-up window that appears over the chat window and
contains web-controls such as text boxes, radio buttons, or drop-down select
boxes. These dialogs can be sent to the chat bot through the responses to user's
input in the ``main.js'' module, and can provide a richer interface for
providing structured data such as selecting subjects from a list or time slots
in a calendar.  The dialogs themselves also have a ``message response''
identifier which will be sent back to the slack bot when the user responds
which is the other primary way users interact with the system.  These response
identifiers provide an easy way to re-order or re-use messages without having to
make large code changes.  % TODO: justify that a bit?
Interestingly, slack only allows 5 controls to be in a single dialogue at one
time.  

% Modules
The last major place for code is the modules folder.  The database access is
handled through the models but helpers are built around complex tasks and stored
in appropriate JS classes in the modules folder.  for example, the actual
database manipulation for tutors happens in the tutor model, but a tutor module
also exists that utilizes that module and adds business logic to it.  


%%% Local Variables:
%%% mode: latex
%%% TeX-master: "../main"
%%% End:
