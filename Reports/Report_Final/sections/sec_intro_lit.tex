% TODO: read more of Kim/few others to find info about peer tutoring
While the merits of peer tutoring are not the major focus of this paper, it is
prudent to give a brief overview of the topic.  Peer tutoring is a type of
tutoring where a student seeks out instruction from another student who has
already studied the subject that the student is interested in learning more on.
There are a number of terms for the tutor in this situation such as ``mentor''
and ``proctor'' but for the purpose of this paper we will simply use tutor to
refer to the student-tutor and student to refer to the student seeking help. \cite{kim}

There are a number of different well-documented benefits to peer tutoring, and
they are not limited simply to the students being tutored.  While the intention
is to improve the students first, the process of teaching another student has
many well-studied benefits for the tutor as well. \cite{kim}

While it is absolutely true that tutors benefit significantly from teaching
their stduents, the obvious goal is to transfer knowledge from the tutor to the
student. It is well documented that students tend to feel more engaged during
one-on-one peer tutoring and that the peer tutoring can give more feedback than
lecture-style learning which can in turn help reduce student anxiety and improve
learning outcomes. \cite{topping}

%%% Local Variables:
%%% mode: latex
%%% TeX-master: "../main"
%%% End:
