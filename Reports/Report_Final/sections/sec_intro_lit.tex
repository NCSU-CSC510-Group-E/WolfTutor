% TODO: read more of Kim/few others to find info about peer tutoring
\subsubsection{Peer Tutoring}
\label{peer-tutoring}
While the merits of peer tutoring are not the major focus of this paper, it is
prudent to give a brief overview of the topic.  Peer tutoring is a type of
tutoring where a student seeks out instruction from another student who has
already studied the subject that the student is interested in learning more on.
There are a number of terms for the tutor in this situation such as ``mentor''
and ``proctor'' but for the purpose of this paper we will simply use tutor to
refer to the student-tutor and student to refer to the student seeking help. \cite{kim}

There are a number of different well-documented benefits to peer tutoring, and
they are not limited simply to the students being tutored.  While the intention
is to improve the students first, the process of teaching another student has
many well-studied benefits for the tutor as well. \cite{kim}

While it is absolutely true that tutors benefit significantly from teaching
their stduents, the obvious goal is to transfer knowledge from the tutor to the
student. It is well documented that students tend to feel more engaged during
one-on-one peer tutoring and that the peer tutoring can give more feedback than
lecture-style learning which can in turn help reduce student anxiety and improve
learning outcomes. \cite{topping}

\subsubsection{Comparisons to Existing Applications}
\label{comparisons}
\paragraph{Tutor Matching}
There are no mainstream applications relevant to automative tutor matching that give the users tutor recommendations. 
Most services only offer a place to find tutors manually, such as the 'Lessons' section on Craiglist. \cite{RefWorks:doc:5abd46a5e4b0770b05a4080c} 
There are services with more detail and structure than Cragislist like Verbling, an online application that contains profiles of Spanish tutors. 
These profiles contain more detailed and personal information about the user, including a photo. \cite{RefWorks:doc:5abd466ce4b0689719ee9277} 
Some services take other approaches. 
The popular tutoring service, Chegg, does not facilitate scheduling or tutoring sessions, 
but instead lets multiple tutors cater to a student's question. \cite{RefWorks:doc:5abd45f7e4b0770b05a407c4}

One source of inspiration for WolfTutor's matching enhancements regarding recommendations 
based on the qualities a tutor possessed was Rate My Professors, a web application that hosts reviews of University professors. 
It allows students that have had the professor to give that professor a review and a star rating useful for
students that have multiple options and want to pick the professor that is most loved. Another great
feature of Rate My Professors is the ability to give the professor tags on their qualities. For example, possible tags
include "participation matters", "clear grading critera", and "lots of homework". \cite{rate-my-prof} These tags allow potential students to
get a better idea for what style the professor uses and what to expect for the class.

Another important source for WolfTutor's enhancements was Amazon, since there currently is 
no go-to tutor recommendation utility.\cite{amazon} Amazon's business model depends on successful product recommendations.
WolfTutor's enhancements essentially attempt to market the tutors as a product to students. Not only is it important
to produce successful recommendations, but the system needs to help promote new products as well to incourage 
growth and competition.

\paragraph{Scheduling Systems}
\label{sec:comparisons:scheduling}
The application of WolfTutor is not actually specific to tutoring entirely.
It can reasonably be compared to any scheduling system.
For example, the appointment scheduling tool within NC State University's epack system functions similarly.
First you pick from an available list of appointment types. Then, you can opt to filter your search further by location, names of possible appointees, time of day, etc.
The tool will show available appointment time slots a user can book just like WolfTutor.
This type of scheduling mechanism is commonly used by other services as well e.g. medical facilities with multiple practitioners, personal trainers, life coaches, etc.




%%% Local Variables:
%%% mode: latex
%%% TeX-master: "../../../main"
%%% End:


%%% Local Variables:
%%% mode: latex
%%% TeX-master: "../main"
%%% End:
