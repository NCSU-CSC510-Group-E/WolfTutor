% TODO: incorporate some graphs of the system before and the system now
Wolftutor has two types of users: students and tutors. Users are automatically
students at signup, but can choose to become a tutor after following a simple
prompt and giving some information about the topics they are interested in, the
times they are available, and some academic information. After enrolling in the
system, the user's name, email, and phone number will be taken from slack and
stored if available.

\subsubsection{Original Student Use Cases}
\label{sec:student-use-cases}

\begin{enumerate}
\item Find a tutor by from a complete list of all available tutors after
  selecting a subject from an existing list
\item See the reviews for at tutor
\item Book a tutor by choosing one of WolfTutor's defined 30 minute slots
  created from the availability given by the tutor
\item Review a tutor from their last session (students have until the end of the
  day to review their tutor)
\item View active reservations
\item View reward point balance (points are used to schedule sessions; 100
  points are given at signup)
  \item Buy more points
\end{enumerate}

\subsubsection{Original Tutor Use Cases}
\label{sec:tutor-use-cases}

\begin{enumerate}
  \item Become a tutor. WolfTutor only allows one major and one degree per tutor, but a tutor can represent multiple subjects. Tutor self-determines points pay rate and availability. Both apply to all subjects.
  \item View tutoring subjects given to WolfTutor
  \item View tutoring availability given to WolfTutor
  \item Redeem points commerically
\end{enumerate}
%%% Local Variables:
%%% mode: latex
%%% TeX-master: "../main"
%%% End:
