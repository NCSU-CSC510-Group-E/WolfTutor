% What do we want to do?
% - what is the goal?
% - Why is that our goal?
% - What do we propose?
% - Why does that solve the problem we are setting out to solve?

Given the surveying discussed above, it seems clear that what students
in our class want is a way to more easily match with competent
tutors.  To that end, WolfTutor currenly does very little in terms of
matching.  It provides students with the ratings of the available
tutors, which is a good first step, but it does nothing to help
organize the available tutors and prioritize them to make searching
sutdents' lives easier.  

WolfTutor also has a significant amount of data about a student's
history with their tutors, but it does fairly little with that
information.  To help improve matching between students and tutors, we
propose to use that historical data during the matching process. To do
that, we will implement a suggestion algorithm that will pull out the
students' previous interactions and use their positive reviews to push
certain tutors higher in the list and their negative reviews to push
other tutors lower on the list.

This will hopefully make it easier for students to match with not only
tutors that are well reviewed and rate well in the subjects they are
interested in, but also encourage students to build longer-term
mentoring relationships with tutors that work well with them.  

Next we will discuss the application's architecture in section
\ref{sec:architecture} and the evaluation plan for this idea \ref{sec:validation}.
%%% Local Variables:
%%% mode: latex
%%% TeX-master: "../../../main"
%%% End:
