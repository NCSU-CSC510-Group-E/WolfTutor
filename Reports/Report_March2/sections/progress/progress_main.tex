As the experience of the previous project, the team also chose the spiral model of software development methodologies. A spiral model is a form of risk-driven development where the most risky parts of a project are identified early and planned into prototypes that can inform the final product. See the breakdown of the prototypes the team chose in section 1.3.

This afforded the team the ability to have regular check-ins and to measure progress against the prototypes. The team will convene two to three times a week to do pair programming and to check progress against the prototypes. When combined with regular check-ins with the TAs, the team shall be able to build prototype programs to perform design feature and functionality.

We also used Gitup project boards to create and track our progress. This method featured the ability to send a notification to the team member to remind and collaborate within the team members. Here is the example image.